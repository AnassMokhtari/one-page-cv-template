% ============================================================
% Modèle de CV 1 page (LaTeX) — Version “placeholders”
% ============================================================

\documentclass[10pt,a4paper]{article}

% ------------------------------------------------------------
% A) MISE EN PAGE (marges serrées pour tenir sur 1 page)
% ------------------------------------------------------------
\usepackage[a4paper,top=0.7cm,bottom=0.7cm,left=0.8cm,right=0.8cm]{geometry}

% ------------------------------------------------------------
% B) POLICES / ENCODAGE (rendu propre et professionnel)
% ------------------------------------------------------------
\usepackage[T1]{fontenc}
\usepackage[utf8]{inputenc}     % garder pour pdflatex ; retirer si lualatex/xelatex
\usepackage{lmodern}
\usepackage{microtype}

% ------------------------------------------------------------
% C) COULEURS + LIENS (personnaliser AccentColor si besoin)
% ------------------------------------------------------------
\usepackage{xcolor}
\colorlet{AccentColor}{blue}    % <--- changer en teal, purple, black, etc.
\usepackage[hidelinks]{hyperref}
\hypersetup{
  colorlinks=true,
  urlcolor=AccentColor
}

% ------------------------------------------------------------
% D) TABLEAUX / COLONNES / OUTILS D’ALIGNEMENT
% ------------------------------------------------------------
\usepackage{array}
\usepackage{tabularx}
\usepackage{ragged2e}
\usepackage{paracol}

% Types de colonnes :
% L = colonne gauche extensible (alignée à gauche, ragged-right)
% D = colonne dates fixe (alignée à droite)
\newcolumntype{L}{>{\RaggedRight\arraybackslash}X}
\newcolumntype{D}{>{\raggedleft\arraybackslash}p{3cm}}

% ------------------------------------------------------------
% E) LISTES / ICÔNES / TITRES DE SECTIONS
% ------------------------------------------------------------
\usepackage{enumitem}
\usepackage{fontawesome5}
\usepackage{titlesec}
\usepackage{tikz} % optionnel ; garder si vous prévoyez des éléments graphiques

% ------------------------------------------------------------
% F) ESPACEMENTS GLOBAUX (style CV)
% ------------------------------------------------------------
\pagestyle{empty}
\setlength{\parindent}{0pt}
\setlength{\parskip}{0pt}

\titlespacing*{\section}{0pt}{6pt}{4pt}

% ------------------------------------------------------------
% G) PETITES COMMANDES UTILES
% ------------------------------------------------------------
\newcommand{\sep}{\hspace{0.6em}{\color{gray}\textbar}\hspace{0.6em}}
\newcommand{\role}[1]{\textbf{#1}}
\newcommand{\company}[1]{\textbf{#1}}
\newcommand{\place}[1]{\textit{#1}}

% ------------------------------------------------------------
% H) CONTRÔLES DE TAILLE (ajuster si vous voulez +/– de densité)
% ------------------------------------------------------------
\newcommand{\BodySize}{\fontsize{7}{11}\selectfont}
\newcommand{\SectionSize}{\fontsize{9}{14}\selectfont}
\newcommand{\ListSize}{\fontsize{7}{10.2}\selectfont}
\newcommand{\ProjectSize}{\fontsize{7}{9.0}\selectfont}
\newcommand{\MetaSize}{\fontsize{7}{10.0}\selectfont}

% Style des titres de section + fine ligne horizontale sous chaque titre
\titleformat{\section}{\SectionSize\bfseries}{}{0pt}{}[\vspace{1pt}\hrule\vspace{1pt}]

% Style global des itemize (compact)
\setlist[itemize]{
  leftmargin=1.2em,
  itemsep=2pt,
  topsep=2pt,
  parsep=0pt,
  partopsep=0pt,
  font=\ListSize
}

% Appliquer la taille du texte après la zone d’en-tête
\newcommand{\ApplyBodySizes}{\BodySize}

% ============================================================
% ===================== MODIFIER ICI (DONNÉES) ================
% ============================================================
% Astuce : ne modifier que cette section lors de la réutilisation du modèle.

% -- Informations d’en-tête
\newcommand{\CVName}{VOTRE NOM COMPLET}
\newcommand{\CVHeadline}{Votre poste / objectif (ex. Ingénieur Systèmes Embarqués)}

\newcommand{\CVEmail}{votre.email@exemple.com}
\newcommand{\CVPhone}{+XXX XX-XXX-XXXX}
\newcommand{\CVLocation}{Ville, Pays}
\newcommand{\CVMobility}{Télétravail / Présentiel ...}
\newcommand{\CVLicense}{(optionnel)}

% -- Liens (optionnel)
\newcommand{\CVLinkedInURL}{https://www.linkedin.com/in/votre-profil/}
\newcommand{\CVLinkedInText}{Identifiant LinkedIn}
\newcommand{\CVGithubURL}{https://github.com/votre-username}
\newcommand{\CVGithubText}{Identifiant GitHub}

% -- Résumé professionnel (court : 3 à 5 lignes)
\newcommand{\CVSummary}{%
Un court résumé professionnel décrivant votre spécialité et vos points forts.
Mentionnez vos compétences clés, vos domaines, et le type de poste visé (ex. C embarqué, CAN/UDS, MATLAB/Simulink, SIL/HIL).
Ajoutez 1 à 2 mots-clés standards/process si pertinent (ex. ISO 26262, notions AUTOSAR).
}

% ============================================================
% ======================= MACROS (BLOCS) ======================
% ============================================================

% Entrée Formation :
% \eduentry{Diplôme / Programme}{Dates}{Établissement / Faculté}
\newcommand{\eduentry}[3]{%
  \begin{tabularx}{\textwidth}{@{}L D@{}}
    \textbf{#1} & {\MetaSize\textit{#2}}\\[-2pt]
    \multicolumn{2}{@{}l@{}}{\MetaSize\textcolor{AccentColor}{#3}}\\
  \end{tabularx}\vspace{-3pt}
}

% Entrée Projet :
% \proj{Titre du projet}{Technologies : ...}
\newcommand{\proj}[2]{%
  \textbf{#1}\\[-1pt]
  {\ProjectSize\textbf{\textit{Technologies :}} #2}\\[-10pt]
}

% ============================================================
% ========================= DOCUMENT ==========================
% ============================================================
\begin{document}
\ApplyBodySizes

% ============================================================
% 1) EN-TÊTE (Nom/Titre à gauche, Contact à droite)
% ============================================================
\noindent
\begin{minipage}[t]{0.705\textwidth}
  \vspace{0pt}
  {\LARGE \textbf{\CVName}}\\[6pt]
  {\large\textbf{\CVHeadline}}
\end{minipage}
\hfill
\begin{minipage}[t]{0.25\textwidth}
  \vspace{0pt}
  \fontsize{7}{9.6}\selectfont
  \raggedright
  Email : \href{mailto:\CVEmail}{\CVEmail}\\
  Téléphone : \CVPhone\\
  Localisation : \CVLocation\\
  Mobilité : \CVMobility\\
  Permis : \CVLicense
\end{minipage}
\vspace{-8pt}

% ============================================================
% 2) RÉSUMÉ PROFESSIONNEL
% ============================================================
\section*{Résumé professionnel}
\CVSummary
\vspace{-4pt}

% ============================================================
% 3) FORMATION (ajouter/supprimer selon besoin)
% ============================================================
\section*{Formation}
\vspace{-2pt}

\eduentry{Nom du diplôme / programme}{Début -- Fin}{Université / Faculté / École}

\eduentry{Nom du diplôme / programme}{Début -- Fin}{Université / Faculté / École}

\eduentry{Nom du diplôme / programme}{Début -- Fin}{Université / Faculté / École}

\vspace{-8pt}

% ============================================================
% 4) PROJETS (ajouter/supprimer selon besoin)
% ============================================================
\section*{Projets}
\vspace{0.5pt}

\proj{Titre du projet 1}
{Tech 1, Tech 2, Tech 3, Outil 1, Outil 2}

\proj{Titre du projet 2}
{Tech 1, Tech 2, Tech 3, Outil 1, Outil 2}

\proj{Titre du projet 3}
{Tech 1, Tech 2, Tech 3, Outil 1, Outil 2}

\proj{Titre du projet 4}
{Tech 1, Tech 2, Tech 3, Outil 1, Outil 2}

\vspace{-6pt}

% ============================================================
% 5) EXPÉRIENCE (1 exemple ; copier/coller pour plus)
% ============================================================
\section*{Expérience}
\vspace{1pt}

{\fontsize{8}{9}\selectfont
\company{Nom de l’entreprise} \sep \place{Ville, Pays}\hfill {\MetaSize\textit{Mois AAAA -- Mois AAAA}} } \\
\role{Intitulé du poste / stage}\\
{\MetaSize{\textbf{Sujet :} Votre sujet / mission (optionnel)}}\\[-11pt]

{\fontsize{7}{9}\selectfont
\begin{itemize}[leftmargin=3em]
  \item Action + ce que vous avez fait + impact mesurable (si possible).
  \item Action + outils/tech utilisés + résultat.
  \item Action + responsabilité + résultat.
  \item Action + documentation / collaboration / tests.
\end{itemize}
}

\vspace{-6pt}

% ============================================================
% 6) COMPÉTENCES TECHNIQUES (regrouper par catégories)
% ============================================================
\section*{Compétences techniques}
\vspace{-0.5pt}

{\fontsize{7}{9}\selectfont
\begin{itemize}
  \item \textbf{Programmation \& Gestion de versions :} Langage 1, Langage 2, Langage 3 \sep Git/GitHub.
  \item \textbf{Embarqué / Matériel :} MCU 1, MCU 2, Périphériques, Protocoles (I2C/SPI/UART), RTOS (optionnel).
  \item \textbf{Réseaux automobiles (optionnel) :} CAN, LIN, UDS, DBC, Outils (CANoe/CANalyzer).
  \item \textbf{Modélisation / Simulation (optionnel) :} MATLAB/Simulink, Simscape, outils MBD.
  \item \textbf{Data / ML (optionnel) :} pandas, NumPy, scikit-learn, etc.
  \item \textbf{Outils :} IDEs, CI, documentation, outils PCB, etc.
\end{itemize}
}

\vspace{-6pt}

% ============================================================
% 7) CONNAISSANCES TECHNIQUES (standards/tests/process)
% ============================================================
\section*{Connaissances techniques}
\vspace{-0.5pt}

{\fontsize{7}{9}\selectfont
\begin{itemize}
  \item \textbf{Standards / Process :} ISO 26262, notions AUTOSAR, Automotive SPICE, MISRA (adapter à votre domaine).
  \item \textbf{Tests \& Validation :} concepts MIL / SIL / HIL, cas de test, reporting, notions de traçabilité.
  \item \textbf{Systèmes / Communication :} timing, contraintes temps réel, trames, diagnostic, etc.
  \item \textbf{Pratiques projet :} cycle en V / Agile, discipline de documentation, conception modulaire.
\end{itemize}
}

\vspace{-6pt}

% ============================================================
% 8) BAS DE PAGE (2 colonnes)
%    Gauche : Certificats, Soft skills, Activités
%    Droite : Intérêts, Langues, Outils, Liens
% ============================================================
\begin{paracol}{2}
\setlength{\columnsep}{0.5cm}

% ---------------- COLONNE GAUCHE ----------------

\section*{Certificats}
\vspace{0pt}

\textbf{Plateforme / Organisme}
{\fontsize{7}{8}\selectfont
\begin{itemize}[leftmargin=3em]
  \item Certificat / Cours 1
  \item Certificat / Cours 2
  \item Certificat / Cours 3
\end{itemize}
}

\vspace{-5pt}

\section*{Compétences interpersonnelles}
Compétence 1 \sep Compétence 2 \sep Compétence 3 \\
Compétence 4 \sep Compétence 5 \sep Compétence 6
\vspace{-4pt}

\section*{Activités extra-scolaires}
\textbf{Club / Association : Rôle}\\[-11pt]
{\fontsize{7}{9}\selectfont
\begin{itemize}[leftmargin=3em]
  \item Activité / responsabilité 1
  \item Activité / responsabilité 2
\end{itemize}
}

% ---------------- COLONNE DROITE ----------------
\switchcolumn
\vspace{0pt}

\section*{Centres d’intérêt}
Intérêt 1 \sep Intérêt 2 \sep Intérêt 3
\vspace{-4pt}

\section*{Langues}
\textbf{Langue 1} (Niveau) \sep
\textbf{Langue 2} (Niveau) \sep
\textbf{Langue 3} (Niveau)
\vspace{-4pt}

\section*{Outils de documentation}
Outil 1 \sep Outil 2 \sep Outil 3 \sep Outil 4
\vspace{-4pt}

\section*{Réseaux sociaux}
{
\setlength{\tabcolsep}{3pt}
\begin{tabular}{@{}l l@{}}
  \faLinkedin & \href{\CVLinkedInURL}{\CVLinkedInText} \\
  \faGithub   & \href{\CVGithubURL}{\CVGithubText} \\
\end{tabular}
}

\end{paracol}

\end{document}
